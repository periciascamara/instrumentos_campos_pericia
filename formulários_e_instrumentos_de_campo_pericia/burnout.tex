\documentclass[a4paper,10pt]{article}
\usepackage[utf8]{inputenc}
\usepackage[T1]{fontenc}
\usepackage[portuguese]{babel}
\usepackage{geometry}
\usepackage{tabularx}
\usepackage{booktabs}
\usepackage{enumitem}
\usepackage{amssymb} % Para os quadradinhos de check
\usepackage{titlesec}
\usepackage{fancyhdr}
\usepackage{pdflscape}

% Configuração das margens
\geometry{left=2cm, right=2cm, top=2cm, bottom=2cm}

% Configuração do cabeçalho
\pagestyle{fancy}
\fancyhf{}
\lhead{\textbf{Avaliação de Bem-estar e Esgotamento Profissional.}}
\rhead{Confidencial}
\cfoot{\thepage}

% Título das seções
\titleformat{\section}{\large\bfseries\uppercase}{}{0em}{}[\titlerule]

\begin{document}

% Cabeçalho do Participante
\begin{center}
    {\LARGE \textbf{Formulário de Avaliação de Burnout}}
\end{center}
\vspace{0.5cm}

\noindent \textbf{Nome/ID:} \underline{\hspace{8cm}} \hfill \textbf{Data:} \_\_/\_\_/\_\_\_\_ \\
\noindent \textbf{Cargo/Função:} \underline{\hspace{8cm}} \hfill \textbf{Setor:} \underline{\hspace{4cm}}

\vspace{0.5cm}
\noindent \textit{Instruções: Este formulário contém dois questionários distintos. Por favor, responda a todas as perguntas com sinceridade, baseando-se em como você tem se sentido em relação ao seu trabalho recentemente. Não existem respostas certas ou erradas.}

% ==========================================================================
% PARTE 1: MBI-GS
% Fonte dos itens: [1] (Anexo V - MBI-GS Brasil) e [5]
% ==========================================================================
\section{Parte I: Inventário Geral (MBI-GS)}

\noindent \textbf{Escala de Frequência:}
\begin{center}
\small
\begin{tabular}{ccccccc}
    \textbf{0} & \textbf{1} & \textbf{2} & \textbf{3} & \textbf{4} & \textbf{5} & \textbf{6} \\
    Nunca & Algumas vezes & Uma vez & Algumas vezes & Uma vez & Algumas vezes & Todos \\
     & ao ano & ao mês & ao mês & por semana & por semana & os dias \\
\end{tabular}
\end{center}

\noindent \textit{Indique o número correspondente à frequência com que você sente cada afirmação:}

%\vspace{0.3cm}

\begin{tabularx}{\textwidth}{ |c|X|c| }
\hline
\textbf{\#} & \textbf{Afirmação} & \textbf{Nota (0-6)} \\
\hline
1 & Sinto-me emocionalmente esgotado com o meu trabalho. & \\[0.3cm] \hline
2 & Sinto-me esgotado no final de um dia de trabalho. & \\[0.3cm] \hline
3 & Sinto-me cansado quando me levanto pela manhã e preciso encarar outro dia de trabalho. & \\[0.3cm] \hline
4 & Posso entender facilmente como meus pacientes/clientes se sentem em relação às coisas & \\[0.3cm] \hline
5 &  Sinto que trato alguns pacientes/clientes como se fossem objetos impessoais & \\[0.3cm] \hline
6 &  Trabalhar com pessoas o dia todo é muito cansativo para mim & \\[0.3cm] \hline
7 &  Lido de forma muito eficaz com os problemas dos meus pacientes/clientes  & \\[0.3cm] \hline
8 & Sinto-me esgotado pelo meu trabalho. & \\[0.3cm] \hline
9 & Sinto que estou influenciando positivamente a vida de outras pessoas através do meu trabalho & \\[0.3cm] \hline
10 & Tornei-me mais insensível com as pessoas desde que comecei este trabalho & \\[0.3cm] \hline
11 & Preocupo-me que este trabalho esteja me endurecendo emocionalmente & \\[0.3cm] \hline
12 & Sinto-me com muita energia & \\[0.3cm] \hline
13 & Sinto-me frustrado com meu trabalho. & \\[0.3cm] \hline
14 & Sinto que estou trabalhando demais no meu trabalho. & \\[0.3cm] \hline
15 & Eu realmente não me importo com o que acontece com alguns pacientes/clientes & \\[0.3cm] \hline
16 & Trabalhar diretamente com pessoas me causa muito estresse. & \\[0.3cm] \hline
17 & Consigo facilmente criar um ambiente descontraído com os meus pacientes/clientes. & \\[0.3cm] \hline
18 & Sinto-me entusiasmado depois de trabalhar em estreita colaboração com os meus pacientes/clientes. & \\[0.3cm] \hline
19 &  Realizei muitas coisas valiosas neste trabalho. & \\[0.3cm] \hline
20 & Sinto que estou no fim da linha (ou no limite das minhas forças). & \\[0.3cm] \hline
21 & No meu trabalho, lido com problemas emocionais com muita calma. & \\[0.3cm] \hline
22 & Sinto que os destinatários me culpam por alguns dos seus problemas. & \\[0.3cm] \hline



\end{tabularx}


\vspace{0.2cm}
\noindent \footnotesize{\textit{*Nota: Este instrumento (MBI) é protegido por direitos autorais. O uso comercial ou em larga escala requer licença da Mind Garden.}}

\newpage




\begin{landscape}
\begin{table}[p]
\centering
\begin{Large}

\begin{tabularx}{\linewidth}{ X c c c c c }
\toprule
\textbf{Perguntas - CBI - Inventário de Copenhagen} & \textbf{Sempre} & \textbf{Freq.} & \textbf{Às vezes} & \textbf{Rara.} & \textbf{Nunca} \\
\midrule
\multicolumn{6}{l}{\textbf{A. Burnout Pessoal}} \\
\hline
1. Com que frequência se sente cansado? & $\square$ & $\square$ & $\square$ & $\square$ & $\square$ \\
2. Com que frequência se sente fisicamente exausto? & $\square$ & $\square$ & $\square$ & $\square$ & $\square$ \\
3. Com que frequência se sente emocionalmente exausto? & $\square$ & $\square$ & $\square$ & $\square$ & $\square$ \\
4. Com que frequência pensa ``Não aguento mais''? & $\square$ & $\square$ & $\square$ & $\square$ & $\square$ \\
5. Com que frequência se sente esgotado? & $\square$ & $\square$ & $\square$ & $\square$ & $\square$ \\
6. Com que frequência se sente fraco e suscetível de adoecer? & $\square$ & $\square$ & $\square$ & $\square$ & $\square$ \\
\midrule
\multicolumn{6}{l}{\textbf{B. Burnout Relacionado ao Trabalho}} \\
\hline
7. Sente-se esgotado no final de um dia de trabalho? & $\square$ & $\square$ & $\square$ & $\square$ & $\square$ \\
8. Sente-se exausto logo pela manhã quando pensa em mais um dia de trabalho? & $\square$ & $\square$ & $\square$ & $\square$ & $\square$ \\
9. Sente que cada hora de trabalho é cansativa para você? & $\square$ & $\square$ & $\square$ & $\square$ & $\square$ \\
10. Tem tempo e energia para a família e amigos durante os tempos de lazer? (Invertido) & $\square$ & $\square$ & $\square$ & $\square$ & $\square$ \\
11. O seu trabalho é emocionalmente esgotante? & $\square$ & $\square$ & $\square$ & $\square$ & $\square$ \\
12. Sente-se frustrado com o seu trabalho? & $\square$ & $\square$ & $\square$ & $\square$ & $\square$ \\
13. Sente-se exausto por causa do seu trabalho? & $\square$ & $\square$ & $\square$ & $\square$ & $\square$ \\
\midrule
\multicolumn{6}{l}{\textbf{C. Burnout Relacionado ao Cliente (ou Alunos/Pacientes/Usuários)}} \\
\hline
14. Você acha difícil trabalhar com clientes? & $\square$ & $\square$ & $\square$ & $\square$ & $\square$ \\
15. Sente que esgota sua energia quando trabalha com clientes? & $\square$ & $\square$ & $\square$ & $\square$ & $\square$ \\
16. Acha frustrante trabalhar com clientes? & $\square$ & $\square$ & $\square$ & $\square$ & $\square$ \\
17. Sente que dá mais do que recebe quando trabalha com clientes? & $\square$ & $\square$ & $\square$ & $\square$ & $\square$ \\
18. Está cansado de trabalhar com clientes? & $\square$ & $\square$ & $\square$ & $\square$ & $\square$ \\
19. Às vezes você se pergunta quanto tempo você será capaz de continuar trabalhando com clientes? & $\square$ & $\square$ & $\square$ & $\square$ & $\square$ \\
\bottomrule
\end{tabularx}

\end{Large}
\end{table}
\end{landscape}



\end{document}
