\documentclass[a4paper,11pt]{article}
\usepackage[utf8]{inputenc}
\usepackage[T1]{fontenc}
\usepackage[portuguese]{babel}
\usepackage[margin=2cm]{geometry}
\usepackage{longtable}
\usepackage{array}
\usepackage{booktabs}
\usepackage{amssymb} % Para os quadrados de checkbox

% Configuração de estilo
\setlength{\parindent}{0pt}
\setlength{\parskip}{1em}

% Comando para checkbox vazio
\newcommand{\checkbox}{\Large$\square$}

\begin{document}

% --- Cabeçalho e Título ---
\begin{center}
    \textbf{\LARGE COPSOQ II – Versão Média} \\
    \vspace{0.2cm}
    \small{(Kristensen, T., 2001) \\ (Tradução e adaptação de Silva, C. et al., 2011)}
\end{center}

\vspace{0.5cm}

% --- BLOCO DE IDENTIFICAÇÃO ---
\noindent\textbf{Identificação do Respondente}

\vspace{0.2cm}
\noindent
\begin{tabular}{@{}p{0.65\textwidth} p{0.30\textwidth}@{}}
    Nome: \hrulefill & Data: \hrulefill \\[1.5em]
    Setor/Departamento: \hrulefill & ID/Código: \hrulefill \\[1.5em]
    Função/Cargo: \hrulefill & Idade: \hrulefill \\
\end{tabular}

\vspace{0.8cm}
\hrule
\vspace{0.5cm}

% --- Instruções ---
\textbf{Instruções:} Das seguintes afirmações, indique com um (X) a que mais se adequa à sua resposta.

% --- Tabela Principal ---
\begin{longtable}{p{10cm} c c c c c}
    
    % --- Cabeçalho da Primeira Página ---
    \toprule
    & \textbf{1} & \textbf{2} & \textbf{3} & \textbf{4} & \textbf{5} \\
    \textbf{Escala Geral (Frequência)} & \tiny{Nunca/} & \tiny{Rara-} & \tiny{Às} & \tiny{Frequente-} & \tiny{Sempre} \\
    & \tiny{Quase nunca} & \tiny{mente} & \tiny{vezes} & \tiny{mente} & \\
    \midrule
    \endfirsthead

    % --- Cabeçalho para páginas seguintes ---
    \multicolumn{6}{l}{\textit{Continuação...}} \\
    \toprule
    & \textbf{1} & \textbf{2} & \textbf{3} & \textbf{4} & \textbf{5} \\
    & \tiny{Nunca/} & \tiny{Rara-} & \tiny{Às} & \tiny{Frequente-} & \tiny{Sempre} \\
    & \tiny{Quase nunca} & \tiny{mente} & \tiny{vezes} & \tiny{mente} & \\
    \midrule
    \endhead

    % --- Rodapé da tabela ---
    \bottomrule
    \endfoot

    % --- PERGUNTAS 1-27 ---
    1. A sua carga de trabalho acumula-se por ser mal distribuída? & $\square$ & $\square$ & $\square$ & $\square$ & $\square$ \\ \addlinespace
    2. Com que frequência não tem tempo para completar todas as tarefas do seu trabalho? & $\square$ & $\square$ & $\square$ & $\square$ & $\square$ \\ \addlinespace
    3. Precisa fazer horas-extra? & $\square$ & $\square$ & $\square$ & $\square$ & $\square$ \\ \addlinespace
    4. Precisa trabalhar muito rapidamente? & $\square$ & $\square$ & $\square$ & $\square$ & $\square$ \\ \addlinespace
    5. O seu trabalho exige a sua atenção constante? & $\square$ & $\square$ & $\square$ & $\square$ & $\square$ \\ \addlinespace
    6. O seu trabalho requer que seja bom a propor novas ideias? & $\square$ & $\square$ & $\square$ & $\square$ & $\square$ \\ \addlinespace
    7. O seu trabalho exige que tome decisões difíceis? & $\square$ & $\square$ & $\square$ & $\square$ & $\square$ \\ \addlinespace
    8. O seu trabalho exige emocionalmente de si? & $\square$ & $\square$ & $\square$ & $\square$ & $\square$ \\ \addlinespace
    9. Tem um elevado grau de influência no seu trabalho? & $\square$ & $\square$ & $\square$ & $\square$ & $\square$ \\ \addlinespace
    10. Participa na escolha das pessoas com quem trabalha? & $\square$ & $\square$ & $\square$ & $\square$ & $\square$ \\ \addlinespace
    11. Pode influenciar a quantidade de trabalho que lhe compete a si? & $\square$ & $\square$ & $\square$ & $\square$ & $\square$ \\ \addlinespace
    12. Tem alguma influência sobre o tipo de tarefas que faz? & $\square$ & $\square$ & $\square$ & $\square$ & $\square$ \\ \addlinespace
    13. O seu trabalho exige que tenha iniciativa? & $\square$ & $\square$ & $\square$ & $\square$ & $\square$ \\ \addlinespace
    14. O seu trabalho permite-lhe aprender coisas novas? & $\square$ & $\square$ & $\square$ & $\square$ & $\square$ \\ \addlinespace
    15. O seu trabalho permite-lhe usar as suas habilidades ou perícias? & $\square$ & $\square$ & $\square$ & $\square$ & $\square$ \\ \addlinespace
    16. No seu local de trabalho, é informado com antecedência sobre decisões importantes, mudanças ou planos para o futuro? & $\square$ & $\square$ & $\square$ & $\square$ & $\square$ \\ \addlinespace
    17. Recebe toda a informação de que necessita para fazer bem o seu trabalho? & $\square$ & $\square$ & $\square$ & $\square$ & $\square$ \\ \addlinespace
    18. O seu trabalho apresenta objectivos claros? & $\square$ & $\square$ & $\square$ & $\square$ & $\square$ \\ \addlinespace
    19. Sabe exactamente quais as suas responsabilidades? & $\square$ & $\square$ & $\square$ & $\square$ & $\square$ \\ \addlinespace
    20. Sabe exactamente o que é esperado de si? & $\square$ & $\square$ & $\square$ & $\square$ & $\square$ \\ \addlinespace
    21. O seu trabalho é reconhecido e apreciado pela gerência? & $\square$ & $\square$ & $\square$ & $\square$ & $\square$ \\ \addlinespace
    22. A gerência do seu local de trabalho respeita-o? & $\square$ & $\square$ & $\square$ & $\square$ & $\square$ \\ \addlinespace
    23. É tratado de forma justa no seu local de trabalho? & $\square$ & $\square$ & $\square$ & $\square$ & $\square$ \\ \addlinespace
    24. Faz coisas no seu trabalho que uns concordam mas outros não? & $\square$ & $\square$ & $\square$ & $\square$ & $\square$ \\ \addlinespace
    25. Por vezes tem que fazer coisas que deveriam ser feitas de outra maneira? & $\square$ & $\square$ & $\square$ & $\square$ & $\square$ \\ \addlinespace
    26. Por vezes tem que fazer coisas que considera desnecessárias? & $\square$ & $\square$ & $\square$ & $\square$ & $\square$ \\ \addlinespace
    27. Com que frequência tem ajuda e apoio dos seus colegas de trabalho? & $\square$ & $\square$ & $\square$ & $\square$ & $\square$ \\ \addlinespace
    28. Com que frequência os seus colegas estão dispostos a ouvi-lo(a) sobre os seus problemas de trabalho? & $\square$ & $\square$ & $\square$ & $\square$ & $\square$ \\ \addlinespace
    29. Com que frequência os seus colegas falam consigo acerca do seu desempenho laboral? & $\square$ & $\square$ & $\square$ & $\square$ & $\square$ \\ \addlinespace
    30. Com que frequência o seu superior imediato fala consigo sobre como está a decorrer o seu trabalho? & $\square$ & $\square$ & $\square$ & $\square$ & $\square$ \\ \addlinespace
    31. Com que frequência tem ajuda e apoio do seu superior imediato? & $\square$ & $\square$ & $\square$ & $\square$ & $\square$ \\ \addlinespace
    32. Com que frequência é que o seu superior imediato fala consigo em relação ao seu desempenho laboral? & $\square$ & $\square$ & $\square$ & $\square$ & $\square$ \\ \addlinespace
    33. Existe um bom ambiente de trabalho entre si e os seus colegas? & $\square$ & $\square$ & $\square$ & $\square$ & $\square$ \\ \addlinespace
    34. Existe uma boa cooperação entre os colegas de trabalho? & $\square$ & $\square$ & $\square$ & $\square$ & $\square$ \\ \addlinespace
    35. No seu local de trabalho sente-se parte de uma comunidade? & $\square$ & $\square$ & $\square$ & $\square$ & $\square$ \\ \addlinespace

    % --- Subseção Chefia ---
    \multicolumn{6}{l}{\textbf{Em relação à sua chefia direta até que ponto considera que...}} \\ \addlinespace
    36. Oferece aos indivíduos e ao grupo boas oportunidades de desenvolvimento? & $\square$ & $\square$ & $\square$ & $\square$ & $\square$ \\ \addlinespace
    37. Dá prioridade à satisfação no trabalho? & $\square$ & $\square$ & $\square$ & $\square$ & $\square$ \\ \addlinespace
    38. É bom no planeamento do trabalho? & $\square$ & $\square$ & $\square$ & $\square$ & $\square$ \\ \addlinespace
    39. É bom a resolver conflitos? & $\square$ & $\square$ & $\square$ & $\square$ & $\square$ \\ \addlinespace

    % --- Subseção Local de Trabalho Geral ---
    \multicolumn{6}{l}{\textbf{As questões seguintes referem-se ao seu local de trabalho no seu todo.}} \\ \addlinespace
    40. Os funcionários ocultam informações uns dos outros? & $\square$ & $\square$ & $\square$ & $\square$ & $\square$ \\ \addlinespace
    41. Os funcionários ocultam informação à gerência? & $\square$ & $\square$ & $\square$ & $\square$ & $\square$ \\ \addlinespace
    42. Os funcionários confiam uns nos outros de um modo geral? & $\square$ & $\square$ & $\square$ & $\square$ & $\square$ \\ \addlinespace
    43. A gerência confia nos seus funcionários para fazerem o seu trabalho bem? & $\square$ & $\square$ & $\square$ & $\square$ & $\square$ \\ \addlinespace
    44. Confia na informação que lhe é transmitida pela gerência? & $\square$ & $\square$ & $\square$ & $\square$ & $\square$ \\ \addlinespace
    45. A gerência oculta informação aos seus funcionários? & $\square$ & $\square$ & $\square$ & $\square$ & $\square$ \\ \addlinespace
    46. Os conflitos são resolvidos de uma forma justa? & $\square$ & $\square$ & $\square$ & $\square$ & $\square$ \\ \addlinespace
    47. As sugestões dos funcionários são tratadas de forma séria pela gerência? & $\square$ & $\square$ & $\square$ & $\square$ & $\square$ \\ \addlinespace
    48. O trabalho é igualmente distribuído pelos funcionários? & $\square$ & $\square$ & $\square$ & $\square$ & $\square$ \\ \addlinespace
    49. Sou sempre capaz de resolver problemas, se tentar o suficiente. & $\square$ & $\square$ & $\square$ & $\square$ & $\square$ \\ \addlinespace
    50. É-me fácil seguir os meus planos e atingir os meus objectivos. & $\square$ & $\square$ & $\square$ & $\square$ & $\square$ \\ \addlinespace

    % --- MUDANÇA DE ESCALA (Q51-60) ---
    \multicolumn{6}{c}{\textbf{Escala: 1-Nada/quase nada 2- um pouco  3 moderatamente - 4 muito 5-Extremamente}} \\ \midrule
    51. O seu trabalho tem algum significado para si? & $\square$ & $\square$ & $\square$ & $\square$ & $\square$ \\ \addlinespace
    52. Sente que o seu trabalho é importante? & $\square$ & $\square$ & $\square$ & $\square$ & $\square$ \\ \addlinespace
    53. Sente-se motivado e envolvido com o seu trabalho? & $\square$ & $\square$ & $\square$ & $\square$ & $\square$ \\ \addlinespace
    54. Gosta de falar com os outros sobre o seu local de trabalho? & $\square$ & $\square$ & $\square$ & $\square$ & $\square$ \\ \addlinespace
    55. Sente que os problemas do seu local de trabalho são seus também? & $\square$ & $\square$ & $\square$ & $\square$ & $\square$ \\ \addlinespace
    
    \multicolumn{6}{l}{\textbf{Em relação ao seu trabalho em geral, quão satisfeito está com...}} \\ \addlinespace
    56. As suas perspectivas de trabalho? & $\square$ & $\square$ & $\square$ & $\square$ & $\square$ \\ \addlinespace
    57. As condições físicas do seu local de trabalho? & $\square$ & $\square$ & $\square$ & $\square$ & $\square$ \\ \addlinespace
    58. A forma como as suas capacidades são utilizadas? & $\square$ & $\square$ & $\square$ & $\square$ & $\square$ \\ \addlinespace
    59. O seu trabalho de uma forma global? & $\square$ & $\square$ & $\square$ & $\square$ & $\square$ \\ \addlinespace
    60. Sente-se preocupado em ficar desempregado? & $\square$ & $\square$ & $\square$ & $\square$ & $\square$ \\ \addlinespace

    % --- MUDANÇA DE ESCALA (Q62-64) ---
    \multicolumn{6}{l}{\textbf{Modo como o trabalho afecta a vida privada:}} \\ 
    \multicolumn{6}{c}{\small{(Escala: 1-Nada/quase nada 2- um pouco  3 moderatamente - 4 muito 5-Extremamente)}} \\ \addlinespace
    62. Sente que o seu trabalho lhe exige muita energia que acaba por afectar a sua vida privada negativamente? & $\square$ & $\square$ & $\square$ & $\square$ & $\square$ \\ \addlinespace
    63. Sente que o seu trabalho lhe exige muito tempo que acaba por afectar a sua vida privada negativamente? & $\square$ & $\square$ & $\square$ & $\square$ & $\square$ \\ \addlinespace
    64. A sua família e os seus amigos dizem-lhe que trabalha demais? & $\square$ & $\square$ & $\square$ & $\square$ & $\square$ \\ \addlinespace

    % --- MUDANÇA DE ESCALA (Q65-72) ---
    \multicolumn{6}{l}{\textbf{Com que frequência durante as últimas 4 semanas sentiu...}} \\
    \multicolumn{6}{c}{\small{(1-Nunca/quase nunca 2-raramente  3- as vezes  4- frequentemente 5-Sempre)}} \\ \addlinespace
    65. Dificuldade a adormecer? & $\square$ & $\square$ & $\square$ & $\square$ & $\square$ \\ \addlinespace
    66. Acordou várias vezes durante a noite e depois não conseguia adormecer novamente? & $\square$ & $\square$ & $\square$ & $\square$ & $\square$ \\ \addlinespace
    67. Fisicamente exausto? & $\square$ & $\square$ & $\square$ & $\square$ & $\square$ \\ \addlinespace
    68. Emocionalmente exausto? & $\square$ & $\square$ & $\square$ & $\square$ & $\square$ \\ \addlinespace
    69. Irritado? & $\square$ & $\square$ & $\square$ & $\square$ & $\square$ \\ \addlinespace
    70. Ansioso? & $\square$ & $\square$ & $\square$ & $\square$ & $\square$ \\ \addlinespace
    71. Triste? & $\square$ & $\square$ & $\square$ & $\square$ & $\square$ \\ \addlinespace
    72. Falta de interesse por coisas quotidianas? & $\square$ & $\square$ & $\square$ & $\square$ & $\square$ \\ \addlinespace

    % --- MUDANÇA DE ESCALA (Q73-76) ---
    \multicolumn{6}{l}{\textbf{Nos últimos 12 meses, no seu local de trabalho:}} \\
    \multicolumn{6}{c}{\small{(1-Nunca/quase nunca 2-raramente  3- as vezes  4- frequentemente 5-Sempre)}} \\ \addlinespace
    73. Tem sido alvo de insultos ou provocações verbais? & $\square$ & $\square$ & $\square$ & $\square$ & $\square$ \\ \addlinespace
    74. Tem sido exposto a assédio sexual indesejado? & $\square$ & $\square$ & $\square$ & $\square$ & $\square$ \\ \addlinespace
    75. Tem sido exposto a ameaças de violência? & $\square$ & $\square$ & $\square$ & $\square$ & $\square$ \\ \addlinespace
    76. Tem sido exposto a violência física? & $\square$ & $\square$ & $\square$ & $\square$ & $\square$ \\ \addlinespace

\end{longtable}

% --- QUESTÃO 61 (SAÚDE) ---
\vspace{0.5cm}
\noindent \textbf{61. Em geral, sente que a sua saúde é:}
\begin{center}
    \begin{tabular}{ccccc}
        \toprule
        Excelente & Muito boa & Boa & Razoável & Deficitária \\
        $\square$ & $\square$ & $\square$ & $\square$ & $\square$ \\
        \bottomrule
    \end{tabular}
\end{center}

\vspace{1cm}
\begin{center}
    \textit{Obrigado pela sua colaboração.}
\end{center}

\end{document}